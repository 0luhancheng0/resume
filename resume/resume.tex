% !TEX program = xelatex

\documentclass{resume}
\usepackage{hyperref}
\hypersetup{
    colorlinks=true,
    linkcolor=blue,
    filecolor=magenta,
    urlcolor=gray,
}
\begin{document}
\pagenumbering{gobble} % suppress displaying page number

\name{Luhan Cheng}

\basicInfo{
  \email{0lh.cheng0@gmail.com} \textperiodcentered\
  \phone{(+61) 451270129} \textperiodcentered\
  \linkedin[luhan-cheng]{https://www.linkedin.com/in/luhan-cheng-ab14b0139/}
  \github[0luhancheng0]{https://www.github.com/0luhancheng0}
}

\section{\faBriefcase\ Profile}

As a computer science (Honours) graduate, I'm interested in applying computational approach to scientific problems that have significant impact upon
  our society.

I have experience in variety of topics including High-performance computing, machine learning, numerical optimization, game theory and multi-agent reinforcement learning.

I recently joined the PhD program at Centre for Information Resilience (CIRES), Swinburne University, working on applying natrual language processing and
  machine learning to create a pathway for automated knowledge mining from scientific articles and invention patents

\section{\faGraduationCap\ Education}


\datedsubsection{\textbf{Swinburne University}, Melbourne}{Nov 2022 -- Present}
\textit{PhD Candidate}

This PhD project will involve the application of natural language processing and machine learning algorithms
  to identify patents and research papers that are semantically similar to a given query document,
  as well as topic modelling to identify technology gaps.

\datedsubsection{\textbf{Monash University}, Melbourne}{Feb 2020 -- Nov 2020}
\textit{Bachelor of Computer Science (Honours)}

I use game theory and reinforcement learning to design algorithms whose main property is emergent cooperation and coordination.
Thesis title: ``Reinforcement Learning with Inequity Averse Preferences for Cooperation and Coordination''.
GPA3.5/4 WAN79.

\datedsubsection{\textbf{Monash University}, Melbourne}{Feb 2018 -- Nov 2019}
\textit{Bachelor of Computer Science}

Been Awarded with Faculty of IT merit scholarship. GPA3.5/WAN79
\datedsubsection{\textbf{Monash College}, Melbourne}{Feb 2017 -- Nov 2017}
\textit{Diploma of Information Technology}

\section{\faUsers\ Work\ Experience}

\datedsubsection{\textbf{Faculty of Information Technology}, Monash University, Melbourne}{Jun 2021 -- Nov 2022}
\role{Teaching Associate}

I was working as tutor for the following units

\begin{itemize}
  \item \href{https://handbook.monash.edu/2022/units/FIT3143?year=2022}{FIT3143} - Parallel Computing
  \item \href{https://handbook.monash.edu/2022/units/FIT2004?year=2022}{FIT2004} - Algorithms and Data Structure
  \item \href{https://handbook.monash.edu/2022/units/FIT2014?year=2022}{FIT2014} - Theory of Computation
  \item \href{https://handbook.monash.edu/2021/units/FIT2102?year=2021}{FIT2102} - Programming Paradigms

\end{itemize}


\datedsubsection{\textbf{Monash eResearch Center}, Monash University, Melbourne}{Jun 2019 -- Sep 2022}
\role{Junior High Performace Computing Consultant (HEW level 6)}

I undertake the responsibilities of managing \href{https://www.monash.edu/researchinfrastructure/eresearch/capabilities/compute}{high performance clusters} at Monash University.
Including providing user support, conducting system analysis, design and maintenance as required.


During my time at MeRC, I developed
\begin{itemize}
  \item Slurm job submission monitoring tools using on top of the PySlurm
  \item Continuous integration pipeline for automatic cluster provision tools using Ansible and OpenStack.
  \item Automated testing pipeline for the software stack.
\end{itemize}

I'm in collaboration with MCEM (Monash Center of Electron Microscopy) designing optimized data processing pipeline for large-volume image datasets.

\datedsubsection{\textbf{Commonwealth Scientific and Industrial Research Organization}, Clayton, Melbourne}{Nov 2018 -- Feb 2019}
\role{Vacation Student}{Individual Project}

I was working on a deep-learning project that applies and integrates state-of-the-art bayesian neural network uncertainty estimation method to
the crystallography image classification pipeline.

\section{\faCogs\ Skills}
\begin{itemize}[parsep=0.5ex]


  \item \textbf{Programming Languages}
  \begin{itemize}
      \item Experienced: python, bash, C
      \item Familiar: Java, SQL, TypeScript/JavaScript, R, Haskell, Minizinc, MIPS, Matlab, Mathematica
  \end{itemize}
  \item \textbf{Machine Learning}
  \begin{itemize}
    \item PyTorch, TensorFlow, JAX
  \end{itemize}
  \item \textbf{High Performance Computing/Cloud Computing}
  \begin{itemize}
    \item Linux system administration
    \item Container orchestration using Singularity and Docker
    \item Experience with deploying cloud infrastructure using terraform
    \item System provisioning using Ansible
  \end{itemize}

\end{itemize}

\section{\faTrophy\ Competition}
\datedsubsection{\textbf{Student Cluster Competition}}{Nov 2018}
Participated in \href{https://www.studentclustercompetition.us/}{student cluster competition} in Supercomputing conference at Dallas, Taxes 2018, as member of
Monash university team, which was the only team from Australia.

I was responsible for optimizing High-Performance Conjugate Gradient (HPCG) benchmark.


\section{\faGroup\ Extra-Curricular\ Activity}


\datedsubsection{\textbf{Monash DeepNeuron}}{Jan 2020 -- Dec 2020}
\role{HPC Branch Lead}

My responsibilities as HPC branch lead at Monash DeepNeuron involves:
\begin{enumerate}
  \item Managing day-to-day activities in HPC branch, including leading the weekly training, providing technical advice and liaison with researchers.
  \item Coordinating with the rest of executive team to organize workshops and delivering public presentations/training events.
\end{enumerate}

In the technical side, I'm responsible for developing reproducible, scalable and robust ML training pipeline for
\href{https://www.deepneuron.org/projects}{Microscopium project} which aims to help researchers explore large collections of images,
specifically in high content screening using generative models.


\datedsubsection{\textbf{Monash Data Fluency}}{Jul 2019 — Feb 2020}
\role{Co-Instructor}{}

I helped \href{https://www.monash.edu/data-fluency}{Monash Data Fluency}
delivered numerous workshops including
\begin{itemize}
  \item Introduction to High-performance Computing and Linux
  \item Introduction to TensorFlow and Deep Learning
  \item Deep Learning for Natural Language Processing
\end{itemize}


\section{\faLanguage\ Additional skills}

I speak native Mandarin and proficient English. I'm also a 4 dan (amateur) Go player.

\end{document}
